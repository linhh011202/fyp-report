\section{Authentication Flows}
\label{sec:auth_flows}

This section describes the detailed authentication flows for each supported method.

\subsection{Email/Password Authentication}

The traditional login flow uses Passport's LocalStrategy for credential validation. Figure~\ref{fig:auth_login_flow} shows the sequence.

\begin{figure}[!ht]
\centering
\includegraphics[width=0.75\textwidth]{images/auth_login_flow.png}
\caption{Email/Password Login Flow}
\label{fig:auth_login_flow}
\end{figure}

The login process:
\begin{enumerate}
    \item Client sends credentials to \texttt{POST /auth/login}.
    \item LocalAuthGuard triggers LocalStrategy validation.
    \item AuthService queries MongoDB and compares password hashes using bcrypt.
    \item If valid, existing tokens are revoked (single session enforcement).
    \item New JWT is generated and stored in UserTokens collection.
    \item Response includes access token, subscription end date, and verification status.
\end{enumerate}

\subsection{Google OAuth 2.0}

Google SSO uses Passport's OAuth 2.0 strategy. Figure~\ref{fig:auth_google_flow} illustrates the flow.

\begin{figure}[!ht]
\centering
\includegraphics[width=0.8\textwidth]{images/auth_google_flow.png}
\caption{Google OAuth 2.0 Flow}
\label{fig:auth_google_flow}
\end{figure}

\subsubsection{Google Strategy Configuration}

\begin{lstlisting}[language=TypeScript, caption={GoogleStrategy Implementation}]
@Injectable()
export class GoogleStrategy extends 
  PassportStrategy(Strategy, 'google') {
  constructor(@Inject(authConfig.KEY) private auth) {
    super({
      clientID: auth.googleClientId,
      clientSecret: auth.googleClientSecret,
      callbackURL: auth.googleCallbackURL,
      scope: ['email', 'profile'],
    });
  }

  async validate(req, accessToken, _refreshToken, 
    profile, done) {
    const user = {
      email: profile.emails[0].value,
      firstName: profile.name.givenName,
      lastName: profile.name.familyName,
    };
    done(null, user);
  }
}
\end{lstlisting}

\subsubsection{Callback Handler Logic}

The callback determines whether to redirect to complete-signup (new user) or sign-in (existing user):

\begin{lstlisting}[language=TypeScript, caption={Google Callback Handler}]
@Get('callback')
@UseGuards(GoogleAuthGuard)
async googleAuthRedirect(@Req() req, @Res() res) {
  const result = await this.authService.checkUserByEmail(
    req.user.email, userAgent, ipAddress
  );

  if (result.requiresPassword) {
    // New user - generate temporary token
    const tempToken = await this.authService
      .generateTemporaryToken(result.email);
    // Redirect to /complete-signup
  } else if (result.accessToken) {
    // Existing user - store token and redirect
  }
}
\end{lstlisting}

\subsection{Apple Sign-In}

Apple Sign-In uses a POST callback with the identity token. Figure~\ref{fig:auth_apple_flow} shows the flow.

\begin{figure}[!ht]
\centering
\includegraphics[width=0.8\textwidth]{images/auth_apple_flow.png}
\caption{Apple Sign-In Flow}
\label{fig:auth_apple_flow}
\end{figure}

\subsubsection{Token Verification}

\begin{lstlisting}[language=TypeScript, caption={Apple Token Verification}]
@Post('callback')
async appleAuthCallback(@Body() body, @Res() res) {
  let appleResponse;
  try {
    // Primary: use apple-signin-auth library
    appleResponse = await appleSignin.verifyIdToken(
      body.id_token, 
      { audience: process.env.APPLE_CLIENT_ID }
    );
  } catch (error) {
    // Fallback: manual verification
    appleResponse = await this.authService
      .verifyAppleToken(body.id_token);
  }

  const result = await this.authService
    .checkUserByEmail(appleResponse.email);
  // Handle new user or existing user...
}
\end{lstlisting}

\subsection{Comparison of SSO Methods}

\begin{table}[!ht]
\centering
\begin{tabular}{|l|p{4.5cm}|p{4.5cm}|}
\hline
\textbf{Aspect} & \textbf{Google OAuth} & \textbf{Apple Sign-In} \\
\hline
Flow Type & OAuth 2.0 redirect & POST with id\_token \\
\hline
Guard & GoogleAuthGuard (Passport) & None (manual verification) \\
\hline
Verification & Passport strategy & apple-signin-auth library \\
\hline
Email Privacy & Direct email provided & May use relay email \\
\hline
\end{tabular}
\caption{Google vs Apple SSO Comparison}
\label{tab:sso_comparison}
\end{table}
