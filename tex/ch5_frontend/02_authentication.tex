\section{Authentication Frontend}
\label{sec:frontend_auth}

The frontend authentication system supports three login methods: email/password, Google OAuth, and Apple Sign-In.

\subsection{Auth Service Architecture}

The \texttt{AuthService} manages all authentication operations and token storage.

\begin{table}[H]
\centering
\begin{tabular}{|l|p{6cm}|}
\hline
\textbf{Method} & \textbf{Description} \\
\hline
\texttt{login()} & Email/password authentication \\
\hline
\texttt{register()} & New account registration \\
\hline
\texttt{initiateGoogleLogin()} & Redirect to Google OAuth \\
\hline
\texttt{initiateAppleLogin()} & Apple SDK sign-in flow \\
\hline
\texttt{refreshToken()} & Refresh expired JWT \\
\hline
\texttt{storeAuthData()} & Store token in localStorage \\
\hline
\texttt{clearAuthData()} & Clear token on logout \\
\hline
\end{tabular}
\caption{Auth Service Methods}
\label{tab:auth_service_methods}
\end{table}

\subsection{Token Storage}

Authentication data is stored in localStorage for persistence:

\begin{lstlisting}[language=TypeScript, caption={LocalStorage Token Schema}]
{
  accessToken: "eyJhbGciOiJIUzI1NiIs...",
  jwt: "eyJhbGciOiJIUzI1NiIs...",   // backward compat
  subscriptionEnd: "1737244800000",  // Unix timestamp
  isVerified: "true"
}
\end{lstlisting}

\subsection{Email/Password Login}

\begin{lstlisting}[language=TypeScript, caption={Login Implementation}]
async login({ email, password }) {
  const response = await http.post('/auth/login', { 
    email, password 
  })
  
  const { accessToken, subscriptionEnd, isVerified } = 
    response.data
  
  this.storeAuthData(accessToken, subscriptionEnd, isVerified)
  return response.data
}
\end{lstlisting}

\subsection{OAuth Integration}

\subsubsection{Google OAuth}

Google OAuth uses redirect-based authentication:

\begin{lstlisting}[language=TypeScript, caption={Google OAuth Initiation}]
initiateGoogleLogin() {
  const frontendUrl = encodeURIComponent(window.location.origin)
  const redirectUrl = `${API_URL}/auth/google/login` + 
    `?redirect=${frontendUrl}/sign-in`
  
  window.location.href = redirectUrl
}
\end{lstlisting}

\subsubsection{Apple Sign-In}

Apple Sign-In uses the Apple SDK:

\begin{lstlisting}[language=TypeScript, caption={Apple Sign-In Initialization}]
initializeAppleSignIn(clientId) {
  if (!window.AppleID) {
    console.warn('Apple SDK not loaded')
    return false
  }

  window.AppleID.auth.init({
    clientId: clientId || process.env.VUE_APP_APPLE_CLIENT_ID,
    scope: 'name email',
    redirectURI: `${API_URL}/auth/apple/callback`,
    usePopup: false
  })
  return true
}

async initiateAppleLogin() {
  await window.AppleID.auth.signIn()
}
\end{lstlisting}

\subsection{Complete Signup Flow}

New OAuth users must complete signup by setting a password:

\begin{lstlisting}[language=TypeScript, caption={Complete Signup Implementation}]
async setPassword(temporaryToken, password, fullname) {
  const response = await http.post('/auth/add-info', {
    token: temporaryToken,
    password,
    fullname
  })
  
  const { accessToken, subscriptionEnd, isVerified } = 
    response.data
  this.storeAuthData(accessToken, subscriptionEnd, isVerified)
  
  localStorage.removeItem('temporaryToken')
  return response.data
}
\end{lstlisting}

\subsection{Auth Mixin}

The auth mixin provides authentication state to components:

\begin{lstlisting}[language=TypeScript, caption={Auth Mixin Usage}]
import { authMixin } from '@/mixins/auth.mixin'

export default {
  mixins: [authMixin],
  // Component now has access to:
  // - this.isAuthenticated
  // - this.accessToken
  // - this.subscriptionEnd
  // - this.logout()
  // - this.verifyAndRefreshToken()
}
\end{lstlisting}

The mixin automatically starts token verification on mount and cleans up on destroy.
